\documentclass{report}

\usepackage{hyperref}
\usepackage{amsmath}

\usepackage{algorithm2e}
\usepackage{listings} % Für Source code
\usepackage{color}
\definecolor{scgreen}{rgb}{0,0.6,0}
\definecolor{scgray}{rgb}{0.6,0.6,0.6}
\definecolor{scpurple}{rgb}{0.6,0,0.6}
\definecolor{orange}{rgb}{1,0.5,0.4}

\lstset{
	backgroundcolor=\color{white},
	basicstyle=\normalsize,
	breaklines=true,
	captionpos=b,
	commentstyle=\color{scpurple}
	frame=single,
	keepspaces=true,
	keywordstyle=\color{scgreen},
	morekeywords={socket},
	numbers=left,
	numbersep=10pt,
	numberstyle=\tiny\color{scgray},
	showspaces=false,
	showstringspaces=false,
	showtabs=false,
	stepnumber=1,
	stringstyle=\color{scpurple},
	tabsize=4
}

\begin{document}
	
	\chapter{Mathematik}
	Sei $\epsilon$ beliebig, aber fest, dann gilt:
	$$ \forall x \in X: x \leq \epsilon $$
	\begin{equation} 
		\alpha \beta \gamma \Gamma \delta \Delta \epsilon \varepsilon \Theta \theta \vartheta \Phi \phi \varphi
	\end{equation}
	\begin{equation}
		\sin \cos \arcsin \sinh \vec{A}
		\rightarrow \leftarrow \Leftrightarrow \top \bot \emptyset
	\end{equation}
	\begin{equation} \tag{2}
		\neg \exists \forall \subset \supset \in \notin \land \lor
		\label{formel3}
	\end{equation}
	\\
	Erklärung: Das Ergebnis in Formel \ref{formel3} ist beachtlich.
	
	\section{Formeln}
	Limes:
	$$ \lim_{x_{i} \to 0} \frac{1}{x^{20}} = \infty $$
	\\ Wurzel:
	$$ \sqrt[n]{a} $$
	\\ Summe:
	$$ \sum_{i=1}^{\infty} 1 = \infty $$
	\\ Integral:
	$$ \int_{i=0}^{\infty} f(x) $$
	
	\section{Klammern}
	$$ (abc) $$
	\\
	$$ \left (\int_{i=0}^{\infty} f(x) dx \middle | x > 0 \right) $$ 
	% \middle | = Großer Senkrechter Strich
	\\
	$$ ( \big( \Big( \bigg( \Bigg( $$
	
	\section{Matrizen und Vektoren}
	
	$$ A^{m,n} = \begin{pmatrix}
		a_{1,1} & a_{1,2} & \cdots & a_{1,n} \\
		a_{2,1} & a_{2,2} & \cdots & a_{2,n} \\
		\vdots 	& \vdots  & \ddots & \vdots  \\
		a_{m,1} & a_{m,2} & \cdots & a_{m,n}
	\end{pmatrix} $$
	\\
	Vektor:
		$$ V = \begin{pmatrix}
		a_{1} \\
		a_{2}  \\
		\vdots 	\\
		a_{m}
	\end{pmatrix} $$
	\\
	$$ f(x) = \begin{cases}
		x & \quad \text{if } x \leq 0 \\
		~ &  \quad \text{else}
	\end{cases}$$

	\chapter{Informatik}
	\section{Algorithmen}
	\begin{algorithm}
		\KwIn{x}
		\KwOut{Spaghetti}
		v $\gets$ 2 \\
		\While{hungry}{
			\eIf{spaghetti available}{make spaghetti}{eat schinken}
			eat
		}
	\caption{Greedy Algo}
	\end{algorithm}

	\section{Quellcode}
	\begin{lstlisting}[language=Python, frame=single]
		print("Hello World")
	\end{lstlisting}

	Import Basics.py:
	\lstinputlisting[language=Python, frame=single]{Basics.py}
	

\end{document}